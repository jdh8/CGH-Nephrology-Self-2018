\documentclass{beamer}
\usepackage[no-math]{fontspec}
\usepackage{xeCJK}
\setCJKmainfont{Source Han Sans TW}
\hypersetup{colorlinks,linkcolor=}

\usetheme{CambridgeUS}
\title[(Self \textit{et al}, 2018)]{Balanced crystalloids versus saline in noncritically ill adults}
\subtitle{Self \textit{et al}, \textit{N Engl J Med}, 2018}
\author[Chen-Pang He]{何震邦 (Chen-Pang He), Intern}
\date{November 12, 2018}
\institute[CGH]{Cathay General Hospital}

\newcommand*{\solo}[1]{\centering\includegraphics[width=\textwidth, height=0.8\textheight, keepaspectratio]{#1}}

\begin{document}
\maketitle

\section{Background}
\begin{frame}{Background}
    Comparative clinical effects of balanced crystalloids and saline are
    uncertain, particularly in noncritically ill patients cared for outside an
    intensive care unit (ICU).
\end{frame}

\begin{frame}{Saline}
    \begin{itemize}
        \item Most commonly used isotonic crystalloid
        \item Chloride concentration (154 mM) higher than plasma (94--111 mM)
        \item Hyperchloremic metabolic acidosis
            \begin{itemize}
                \item Renal inflammation
                \item Impaired renal perfusion
            \end{itemize}
    \end{itemize}
\end{frame}

\begin{frame}{Balanced crystalloids}
    Similar chloride concentration to plasma
    \begin{itemize}
        \item Lactated Ringer's solution (109 mM)
        \item Plasma-Lyte A (98 mM)
    \end{itemize}
\end{frame}

\section{Methods}
\begin{frame}{SALT-ED trial}
    \begin{itemize}
        \item Saline against Lactated Ringer's or Plasma-Lyte in the Emergency Department
        \item Single-center, pragmatic, unblinded, multiple-crossover trial
        \item Consecutive noncritically ill adults treated with intravenous
            crystalloids in the emergency department before hospitalization
            outside the ICU
    \end{itemize}
\end{frame}

\begin{frame}{Trial population}
    \begin{itemize}
        \item From January 2016 to April 2017
        \item Vanderbilt University Medical Center Adult Emergency Department
        \item Adults ($\ge$ 18 years old)
        \item Received at least 500 ml of intravenous isotonic crystalloids in ED
        \item Subsequently hospitalized outside an ICU
            \begin{itemize}
                \item Seperate trial for patients who were admitted to an ICU
            \end{itemize}
        \item First ED visit [for uniqueness]
    \end{itemize}
\end{frame}

\begin{frame}{Treatment assignments}
    \begin{itemize}
        \item Type of crystalloid was assigned according to calendar month.
        \item Clinicians could choose between balanced crystalloids.
        \item Clinicians and patients were aware of the treatment assignments.
        \item Saline might be used instead of balanced crystalloids for their relative contraindications.
            \begin{itemize}
                \item Hyperkalemia, brain injury
                \item Determined by treatment provider
            \end{itemize}
    \end{itemize}
\end{frame}

\begin{frame}{Outcomes}
    Primary outcome: hospital-free days to day 28

    Secondary outcomes:
    \begin{itemize}
        \item Major adverse kidney events within 30 days
            \begin{itemize}
                \item Death, new renal-replacement therapy, or persistent renal dysfunction
            \end{itemize}
        \item Acute kidney injury of stage 2 or higher
        \item In-hospital death
    \end{itemize}
\end{frame}

\begin{frame}{Schedule of treatment allocation}
    \solo{S1.eps}
\end{frame}

\begin{frame}{Patient enrollment}
    \solo{S2.eps}
\end{frame}

\begin{frame}{Crystalloid volumes administered}
    \solo{S3.eps}
\end{frame}

\section{Results}
\begin{frame}{Results}
    \begin{itemize}
        \item The number of hospital-free days did not differ
            \begin{itemize}
                \item Median 25 days in each group
                \item Odds ratio 0.98 (0.92--1.04)
            \end{itemize}
        \item Balanced crystalloids resulted in a lower incidence of major adverse kidney events within 30 days
            \begin{itemize}
                \item 4.7\% vs 5.6\%
                \item Odds ratio 0.82 (0.70--0.95)
            \end{itemize}
    \end{itemize}
\end{frame}

\begin{frame}{Baseline characteristics}
    \solo{T1a.eps}
\end{frame}

\begin{frame}{Baseline characteristics (cont.)}
    \solo{T1b.eps}
\end{frame}

\begin{frame}{Crystalloids received in ED}
    \solo{T2.eps}
\end{frame}

\begin{frame}{Outcomes}
    \solo{T3.eps}
\end{frame}

\begin{frame}{Serum electrolyte concentrations}
    \solo{F1.eps}
\end{frame}

\begin{frame}{Heterogeneity}
    \solo{F2.eps}
\end{frame}

\begin{frame}{Hospital-free days}
    \solo{S4.eps}
\end{frame}

\begin{frame}{Incidence of MAKE30 by group}
    \solo{S5.eps}
\end{frame}

\begin{frame}{Incidence of MAKE30 by fluid type and volume}
    \solo{S6.eps}
\end{frame}

\section{Discussion}
\begin{frame}{Discussion}
    \begin{itemize}
        \item Treatment with balanced crystalloids did not result in a shorter
            time to hospital discharge, but did result in better secondary
            outcomes.
        \item The lower incidence of MAKE30 in the balanced-crystalloids group
            is consistent with the results of SMART.
        \item 0.9\% difference in the risk of MAKE30
    \end{itemize}
\end{frame}

\begin{frame}{Strength of this trial}
    High adherence to the assigned crystalloid group due to an unblinded,
    pragmatic design in a learning health care system
\end{frame}

\begin{frame}{Limitations}
    \begin{itemize}
        \item Single-center setting
        \item Unblinded design
        \item Outcome ascertainment was limited to the index hospitalization
    \end{itemize}
\end{frame}
\end{document}
